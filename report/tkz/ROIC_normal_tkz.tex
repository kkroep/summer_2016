\begin{figure}[H]
\centering

\usetikzlibrary{shapes,snakes}

\newcommand*{\Vg}{Vg\\ $\color{red}4.5\,V$}
\newcommand*{\Rst}{\textbf{Rst[3]}\\ $\color{blue}reset$} 
\newcommand*{\Res}{Res\\ $\color{red}0.86\,V$}
\newcommand*{\VDD}{VDD3.3\\ $\color{red}3.3\,V$}
\newcommand*{\IN}{$\color{blue}V_{in}$}
\newcommand*{\OUT}{$\color{blue}V_{out}$}
\newcommand*{\VBO}{\color{blue}\textbf{VBO[8]} $\color{red}1.4\,V$}
\newcommand*{\gnd}{gnd\\ $\color{red}0\,V$}
\newcommand*{\C}{$\color{red}450\,fF$}




\tikzstyle{dot} = [draw,shape=circle,fill=black, scale =.3]
\tikzstyle{l_arrow} = [draw,fill = black, regular polygon,regular polygon sides=3, rotate=-90, anchor=south, scale=.5] 
\tikzstyle{l_text} = [anchor=south west]
\tikzstyle{r_arrow} = [draw,fill = black, regular polygon,regular polygon sides=3, rotate=90, anchor=south, scale=.5] 
\tikzstyle{r_text} = [anchor=south east]
\begin{tikzpicture}[scale=1, every node/.style={scale=1}]




\draw[dashed]  (-0.5,5.5) rectangle (0.5,-2);
\node[align=center, anchor=south] at (0,5.5) {voltage\\limiter};

\draw[dashed]  (1.25,5.5) rectangle (5.25,-2);
\node[align=center, anchor=south] at (3.25,5.5) {integrator};

\draw[dashed]  (5.75,5.5) rectangle (9.25,-2);
\node[align=center, anchor=south] at (7.5,5.5) {current mirrors};

%\draw (0,0) to node[nfet]{};

%\draw (0,0) to (mos1.s);
\node(Vg)[nfet, rotate=-90] at (0,2.5) {};
\node[nfet, rotate=-90] (Reset) at (2.5,2) {};
\node[pfet] (CM_H1) at (5,3) {};
\node[nfet] (CM_L1) at (5,-1) {};
\node[nfet] (CM_H2) at (7,3) {};
\node[nfet] (CM_L2) at (7,-1) {};
\node[nfet] (CM_H3) at (9,3) {};
\node[nfet] (CM_L3) at (9,-1) {};



\draw (-1,2.5) node[anchor=east, align=center]{} to (Vg.S);
\draw (Vg.G) |- (0,4.5) node[anchor=south, align=center]{\Vg};
\draw (Vg.B) |- (1,2.5) node[dot]{} |- (CM_H1.G); %top
\draw (1,2.5) |- (1,0.5)  to [C=\C](4,0.5) -| (CM_H1.D);
\draw (1.5,0.5) node[dot]{} -| (1.5,2) |- (Reset.B);
\draw (Reset.G) to (2.5,4.5) node[anchor=south, align=center]{\Rst};
\draw (Reset.D) -| (3.5,0.5) node[dot]{};
\draw (5,0.5) node[dot]{} to (CM_L1.D);
\draw (CM_L1.G) to (4,4.5) node[anchor=south, align=center]{\Res}; 
\draw (CM_L1.S) to (CM_L2.S) to (7,-2.5) node[anchor=north, align=center]{\gnd};
\draw (CM_H1.B) |- (CM_H2.D) to (7,4.5) node[anchor=south, align=center]{\VDD};
\draw (CM_L2.G) |- (4,0) node[dot]{};
\draw (5,0.5) -| (CM_H2.G);
\draw (CM_L1.B) to (CM_L1.S);
\draw (CM_L2.B) to (CM_L2.S);
\draw (CM_H2.B) to (CM_L2.D);
\draw (CM_H3.B) to (CM_L3.D);
\draw (1,3)node[dot]{} |- (8,4) |- (CM_H3.G);
\draw (CM_H3.D) |- (CM_H2.D) node[dot]{};
\draw (CM_L3.S) |- (CM_L2.S) node[dot]{};
\draw (CM_L3.G) |- (6,0) node[dot]{};
\draw (7,1.5)node[dot]{} to (10,1.5) node[anchor=west, align=center]{\OUT};
%\draw (9,1)node[dot]{} to (10,1) node[anchor=west, align=center]{\VBO};

\draw (-2.5,2.5)node[anchor=east, align=center]{\IN} to [R=$20\,M\Omega$](-1,2.5);


\end{tikzpicture}

\caption{Schematic of ROIC channel template}
\label{tkz:schematic_ROIC_template}
\end{figure}
