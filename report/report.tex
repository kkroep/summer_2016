\documentclass{article}
\usepackage[utf8x]{inputenc}
\usepackage{graphicx}
\usepackage{tikz, amsmath, amssymb, bm, color}
\usepackage{geometry}
\usetikzlibrary{calc}
\usetikzlibrary{shapes,arrows}
\usepackage{todonotes}
\usepackage[american]{circuitikz}
\usepackage{pgfplots}
\usepackage{epstopdf}
\usepackage{listings}
\usepackage{subcaption}
\usepackage{mwe}
\usepackage{float}
\usepackage{cleveref}
\usepackage{ragged2e} % For text alignment


\newenvironment{custom_itemize}{
\begin{itemize}
  \setlength{\itemsep}{0pt}
  \setlength{\parskip}{0pt}
  \setlength{\parsep}{0pt}
}{\end{itemize}}


%\usepackage{multicol} % Used for the two-column layout of the document


%\usepackage{bm}


\title{Preethis ROIC analysis}
\author{Kees Kroep 4246373}


\begin{document}
%  \twocolumn[{%
% \begin{@twocolumnfalse}
  \maketitle
%   \end{@twocolumnfalse}
% }]

\section{setup}\label{sec:setup}
\begin{figure}[H]
\centering
\usetikzlibrary{shapes,snakes}
\tikzstyle{dot} = [draw,shape=circle,fill=black, scale =.2]
\tikzstyle{l_arrow} = [draw,fill = black, regular polygon,regular polygon sides=3, rotate=-90, anchor=south, scale=.5] 
\tikzstyle{l_text} = [anchor=south west]
\tikzstyle{r_arrow} = [draw,fill = black, regular polygon,regular polygon sides=3, rotate=90, anchor=south, scale=.5] 
\tikzstyle{r_text} = [anchor=south east]
\begin{tikzpicture}[scale=1.5, every node/.style={scale=1}]


\node[l_text] at (-3,1) {VDD 3.3 (1)};
\node[l_text] at (-3,0) {IN[8] (15)};
\node[l_text] at (-3,-1) {VSUB (16)};
\node[l_text] at (-3,-2) {VDD\_HV (17)};
\node[l_text] at (-3,-3) {GND\_HV (18)};

\node[l_arrow] at (-3,1) {};
\node[l_arrow] at (-3,0) {};
\node[l_arrow] at (-3,-1) {};
\node[l_arrow] at (-3,-2) {};
\node[l_arrow] at (-3,-3) {};

\node[r_text] at (0,1) {(25) gnd};
\node[r_text] at (0,0) {(26) VDD5};
\node[r_text] at (0,-1) {(27) Vg};
\node[r_text] at (0,-2) {(28) Rst[3]};
\node[r_text] at (0,-3) {(29) Rst[1]};
\node[r_text] at (0,-4) {(30) Rst[2]};
\node[r_text] at (0,-5) {(31) Res};
\node[r_text] at (0,-6) {(32) VB0[8]};
\node[r_text] at (0,-7) {(33) out[8]};
\node[r_text] at (0,-8) {(48) gnd};


\node[r_arrow] at (0,1) {};
\node[r_arrow] at (0,0) {};
\node[r_arrow] at (0,-1) {};
\node[r_arrow] at (0,-2) {};
\node[r_arrow] at (0,-3) {};
\node[r_arrow] at (0,-4) {};
\node[r_arrow] at (0,-5) {};
\node[r_arrow] at (0,-6) {};
\node[r_arrow] at (0,-7) {};
\node[r_arrow] at (0,-8) {};
\draw  (-3,2) rectangle (0,-8);




\draw (-3.5,-1) node[ground]{} to (-3,-1);
\draw (-3.5,-3) node[ground]{} to (-3,-3);
\draw (0.5,1) node[ground]{} to (0,1);
\draw (0.5,-8) node[ground]{} to (0,-8);
\draw (0.5,-3) node[ground]{} to (0,-3);
\draw (0.5,-4) node[ground]{} to (0,-4);

\draw (0.5,0.25) node[anchor=south]{$5\,V$} (0.5,0.25) node[tground]{} to (0.5,0)to (0,0); % VDD5
\draw (0.5,-0.75) node[anchor=south]{$4.5\,V$} (0.5,-0.75) node[tground]{} to (0.5,-1)to (0,-1); % Vg
\draw (-3.5,1.25) node[anchor=south]{$3.3\,V$} (-3.5,1.25) node[tground]{} to (-3.5,1)to (-3,1); % VDD3.3
\draw (-3.5,-1.75) node[anchor=south]{$5\,V$} (-3.5,-1.75) node[tground]{} to (-3.5,-2)to (-3,-2); %VDD_HV
\draw (-4.5,0.25) node[anchor=south]{$set$} (-4.5,0.25) node[tground]{} to (-4.5,0)to (-4.5,0); % IN[8]
\draw (0.5,-1.75) node[anchor=south]{$reset$} (0.5,-1.75) node[tground]{} to (0.5,-2)to (0,-2); % Rst[3]
\draw  (0,-5) to[R=$50\,k\Omega$](1.5,-5) to (1.5,-4.75) node[tground]{} (1.5,-4.75) node[anchor=south]{$5\,V$}; %res


\draw (-4.5,0) to [R=$20\,M\Omega$](-3,0);

\end{tikzpicture}
\caption{Schematic of breadbord}
\label{tkz:breadbord}
\end{figure}
\begin{figure}[H]
\centering
\tikzstyle{dot} = [draw,shape=circle,fill=black, scale =.3]
\tikzstyle{l_arrow} = [draw,fill = black, regular polygon,regular polygon sides=3, rotate=-90, anchor=south, scale=.5] 
\tikzstyle{l_text} = [anchor=south west]
\tikzstyle{r_arrow} = [draw,fill = black, regular polygon,regular polygon sides=3, rotate=90, anchor=south, scale=.5] 
\tikzstyle{r_text} = [anchor=south east]
\begin{tikzpicture}[scale=1, every node/.style={scale=1}]

\draw[dashed, color=gray]  (-0.5,5.5) rectangle (0.5,-2);
\fill[color=gray, opacity=.1]  (-0.5,5.5) rectangle (0.5,-2);
\node[align=center, anchor=south] at (0,5.5) {voltage\\limiter};

\draw[dashed, color=gray]  (1.25,5.5) rectangle (5.25,-2);
\fill[opacity=.1, color=gray]  (1.25,5.5) rectangle (5.25,-2);
\node[align=center, anchor=south] at (3.25,5.5) {integrator};



\draw[dashed, color=gray]  (5.75,5.5) rectangle (9.25,-2);
\fill[opacity=.1, color=gray] (5.75,5.5) rectangle (9.25,-2);
\node[align=center, anchor=south] at (7.5,5.5) {current mirrors};

%\draw (0,0) to node[nfet]{};

%\draw (0,0) to (mos1.s);
\node(Vg)[nfet, rotate=-90] at (0,2.5) {};
\node[pfet, rotate=-90] (Reset) at (2.5,2) {};
\node[pfet] (CM_H1) at (5,3) {};
\node[nfet] (CM_L1) at (5,-1) {};
\node[nfet] (CM_H2) at (7,3) {};
\node[nfet] (CM_L2) at (7,-1) {};
\node[nfet] (CM_H3) at (9,3) {};
\node[nfet] (CM_L3) at (9,-1) {};



\draw (-1,2.5) node[anchor=east]{IN[i]} to (Vg.S);
\draw (Vg.G) |- (0,4.5) node[anchor=south]{Vg};
\draw (Vg.B) |- (1,2.5) node[dot]{} |- (CM_H1.G); %top
\draw (1,2.5) |- (1,0.5)  to [C=$C_{fb}$](4,0.5) -| (CM_H1.D);
\draw (1.5,0.5) node[dot]{} -| (1.5,2) |- (Reset.B);
\draw (Reset.G) to (2.5,4.5) node[anchor=south]{Rst[3]};
\draw (Reset.D) -| (3.5,0.5) node[dot]{};
\draw (5,0.5) node[dot]{} to (CM_L1.D);
\draw (CM_L1.G) to (4,4.5) node[anchor=south]{Res}; 
\draw (CM_L1.S) to (CM_L2.S) to (7,-2.5) node[anchor=north]{gnd};
\draw (CM_H1.B) |- (CM_H2.D) to (7,4.5) node[anchor=south]{VDD3.3};
\draw (CM_L2.G) |- (4,0) node[dot]{};
\draw (5,0.5) -| (CM_H2.G);
\draw (CM_L1.B) to (CM_L1.S);
\draw (CM_L2.B) to (CM_L2.S);
\draw (CM_H2.B) to (CM_L2.D);
\draw (CM_H3.B) to (CM_L3.D);
\draw (1,3)node[dot]{} |- (8,4) |- (CM_H3.G);
\draw (CM_H3.D) |- (CM_H2.D) node[dot]{};
\draw (CM_L3.S) |- (CM_L2.S) node[dot]{};
\draw (CM_L3.G) |- (6,0) node[dot]{};
\draw (7,1.5)node[dot]{} to (10,1.5) node[anchor=west]{OUT[i]};
\draw (9,1)node[dot]{} to (10,1) node[anchor=west]{VBO[i]};



\end{tikzpicture}\caption{Schematic of ROIC channel}
\label{tkz:schematic_ROIC}
\end{figure}


\begin{figure}[h]
	\centering
	\includegraphics[width=0.6\linewidth]{fig/P1010158.JPG}
	\caption{setup overview}
	\label{fig:setup_overview}
\end{figure}

\begin{figure}[h]
	\centering
	\includegraphics[width=0.6\linewidth]{fig/P1010159.JPG}
	\caption{close-up}
	\label{fig:close-up}
\end{figure}

\clearpage

\section{Reset mode}
This test addresses the behavior of the circuit in reset mode. \Cref{tkz:schematic_ROIC_reset} shows the measured values during reset mode. Note that the input voltage is $2.4\,V$, which is important when calculating the input current. 

\input{tkz/ROIC_reset-on_tkz}


\section{Integrator}
This section aims to address the performance of the integrator. \Cref{tkz:integrator_test} shows the setup used for this test. Channel 8 was used, so the end of the $100\,M\Omega$ resistor is connected to IN[8], and the probe is connected to OUT[8]. 

\begin{figure}[H]
\centering

\usetikzlibrary{shapes,snakes}

\newcommand*{\Vg}{Vg\\ $\color{red}4.5\,V$}
\newcommand*{\Rst}{\textbf{Rst[3]}\\ $\color{blue}reset$} 
\newcommand*{\Res}{Res\\ $\color{red}0.86\,V$}
\newcommand*{\VDD}{VDD3.3\\ $\color{red}3.3\,V$}
\newcommand*{\IN}{$\color{blue}V_{in}$}
\newcommand*{\OUT}{$\color{blue}V_{out}$}
\newcommand*{\VBO}{\color{blue}\textbf{VBO[8]} $\color{red}1.4\,V$}
\newcommand*{\gnd}{gnd\\ $\color{red}0\,V$}
\newcommand*{\C}{$\color{blue}C$}




\tikzstyle{dot} = [draw,shape=circle,fill=black, scale =.3]
\tikzstyle{l_arrow} = [draw,fill = black, regular polygon,regular polygon sides=3, rotate=-90, anchor=south, scale=.5] 
\tikzstyle{l_text} = [anchor=south west]
\tikzstyle{r_arrow} = [draw,fill = black, regular polygon,regular polygon sides=3, rotate=90, anchor=south, scale=.5] 
\tikzstyle{r_text} = [anchor=south east]
\begin{tikzpicture}[scale=1, every node/.style={scale=1}]




\draw[dashed]  (-0.5,5.5) rectangle (0.5,-2);
\node[align=center, anchor=south] at (0,5.5) {voltage\\limiter};

\draw[dashed]  (1.25,5.5) rectangle (5.25,-2);
\node[align=center, anchor=south] at (3.25,5.5) {integrator};

\draw[dashed]  (5.75,5.5) rectangle (9.25,-2);
\node[align=center, anchor=south] at (7.5,5.5) {current mirrors};

%\draw (0,0) to node[nfet]{};

%\draw (0,0) to (mos1.s);
\node(Vg)[nfet, rotate=-90] at (0,2.5) {};
\node[nfet, rotate=-90] (Reset) at (2.5,2) {};
\node[pfet] (CM_H1) at (5,3) {};
\node[nfet] (CM_L1) at (5,-1) {};
\node[nfet] (CM_H2) at (7,3) {};
\node[nfet] (CM_L2) at (7,-1) {};
\node[nfet] (CM_H3) at (9,3) {};
\node[nfet] (CM_L3) at (9,-1) {};



\draw (-1,2.5) node[anchor=east, align=center]{} to (Vg.S);
\draw (Vg.G) |- (0,4.5) node[anchor=south, align=center]{\Vg};
\draw (Vg.B) |- (1,2.5) node[dot]{} |- (CM_H1.G); %top
\draw (1,2.5) |- (1,0.5)  to [C=\C](4,0.5) -| (CM_H1.D);
\draw (1.5,0.5) node[dot]{} -| (1.5,2) |- (Reset.B);
\draw (Reset.G) to (2.5,4.5) node[anchor=south, align=center]{\Rst};
\draw (Reset.D) -| (3.5,0.5) node[dot]{};
\draw (5,0.5) node[dot]{} to (CM_L1.D);
\draw (CM_L1.G) to (4,4.5) node[anchor=south, align=center]{\Res}; 
\draw (CM_L1.S) to (CM_L2.S) to (7,-2.5) node[anchor=north, align=center]{\gnd};
\draw (CM_H1.B) |- (CM_H2.D) to (7,4.5) node[anchor=south, align=center]{\VDD};
\draw (CM_L2.G) |- (4,0) node[dot]{};
\draw (5,0.5) -| (CM_H2.G);
\draw (CM_L1.B) to (CM_L1.S);
\draw (CM_L2.B) to (CM_L2.S);
\draw (CM_H2.B) to (CM_L2.D);
\draw (CM_H3.B) to (CM_L3.D);
\draw (1,3)node[dot]{} |- (8,4) |- (CM_H3.G);
\draw (CM_H3.D) |- (CM_H2.D) node[dot]{};
\draw (CM_L3.S) |- (CM_L2.S) node[dot]{};
\draw (CM_L3.G) |- (6,0) node[dot]{};
\draw (7,1.5)node[dot]{} to (10,1.5) node[anchor=west, align=center]{\OUT};
%\draw (9,1)node[dot]{} to (10,1) node[anchor=west, align=center]{\VBO};

\draw (-2.5,2.5)node[anchor=east, align=center]{\IN} to [R=$20\,M\Omega$](-1,2.5);


\end{tikzpicture}

\caption{Schematic of ROIC channel template}
\label{tkz:integrator_test}
\end{figure}


One can calculate the expected performance of the integrator using \cref{eq:t_vs_I}. The expected time to discharge the capacitance versus the measured time is plotted in \cref{fig:e_vs_m}. The expected and measured values don't quite match. After playing around with some values I found that assuming a paracitic capacitance of $250\,fF$ gave the best match. The expected performance with added paracitic capacitance versus measured is shown in \cref{fig:e_vs_m_250fF}.

\begin{align}
	q&=C\cdot V\\
	I&=\frac{V_{in}-V_0}{R}\\
	t&=\frac{q}{I}\\
	t&=\frac{CVR}{V_{in}-V_0}\label{eq:t_vs_I}
\end{align}


\begin{figure}
	\centering
	\begin{subfigure}[b]{0.475\textwidth}
	    \centering
	    \includegraphics[width=\textwidth]{fig/vin_vs_time_450fF.eps}
	    \caption[Network2]%
	    {$C=450\,fF$}    
	    \label{fig:e_vs_m_450fF}
	\end{subfigure}
	\hfill
	\begin{subfigure}[b]{0.475\textwidth}  
	    \centering 
	    \includegraphics[width=\textwidth]{fig/vin_vs_time_350fF.eps}
	    \caption[]%
	    {$C=350\,fF$}    
	    \label{fig:e_vs_m_350fF}
	\end{subfigure}
	\vskip\baselineskip
	\begin{subfigure}[b]{0.475\textwidth}   
	    \centering 
	    \includegraphics[width=\textwidth]{fig/vin_vs_time_150fF.eps}
	    \caption[]%
	    {$C=150\,fF$}    
	    \label{fig:e_vs_m_150fF}
	\end{subfigure}
	\quad
	\begin{subfigure}[b]{0.475\textwidth}   
	    \centering 
	    \includegraphics[width=\textwidth]{fig/vin_vs_time_50fF.eps}
	    \caption[]%
	    {$C=50\,fF$}    
	    \label{fig:e_vs_m_50fF}
	\end{subfigure}
	\caption{Expected versus measured charge up times for different input voltages. The input voltage is connected to the input through a resistor of $100\,M\Omega$}
	\label{fig:e_vs_m}
\end{figure}

\begin{figure}
	\centering
	\begin{subfigure}[b]{0.475\textwidth}
	    \centering
	    \includegraphics[width=\textwidth]{fig/vin_vs_time_450fF_250fF.eps}
	    \caption[Network2]%
	    {$C=450\,fF$}    
	    \label{fig:e_vs_m_450fF_250fF}
	\end{subfigure}
	\hfill
	\begin{subfigure}[b]{0.475\textwidth}  
	    \centering 
	    \includegraphics[width=\textwidth]{fig/vin_vs_time_350fF_250fF.eps}
	    \caption[]%
	    {$C=350\,fF$}    
	    \label{fig:e_vs_m_350fF_250fF}
	\end{subfigure}
	\vskip\baselineskip
	\begin{subfigure}[b]{0.475\textwidth}   
	    \centering 
	    \includegraphics[width=\textwidth]{fig/vin_vs_time_150fF_250fF.eps}
	    \caption[]%
	    {$C=150\,fF$}    
	    \label{fig:e_vs_m_150fF_250fF}
	\end{subfigure}
	\quad
	\begin{subfigure}[b]{0.475\textwidth}   
	    \centering 
	    \includegraphics[width=\textwidth]{fig/vin_vs_time_50fF_250fF.eps}
	    \caption[]%
	    {$C=50\,fF$}    
	    \label{fig:e_vs_m_50fF_250fF}
	\end{subfigure}
	\caption{Expected versus measured charge up times for different input voltages. The expected performance is modified by assuming a $250\,fF$ paracitic capacitance. The input voltage is connected to the input through a resistor of $100\,M\Omega$}
	\label{fig:e_vs_m_250fF}
\end{figure}

\end{document}












