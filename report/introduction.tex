\section{Introduction}\label{sec:introduction}

\subsection{Motivation}\label{ssec:motivation}
UV sensors have numerous applications including oberservation during assembly and the oberservation of outer space. One of the challenges of UV sensors is to deal with sensitivity to visible light, which is the main product of solar irradiation. Gallium Nitride (GaN) has intrinsic solar blindness, which makes it a prime candidate for UV sensors. The material can also still be used in solid state devices. There is currently no alternative to this combination of solarblindness and solid state. This research project focusses on the nontrivial task of designing and characterizing a suitable readout circuit for a GaN sensor. The project builds on the work of Padmanabhan et al. \cite{preethi}. who designed a first version of a Readout Integrated Circuit (ROIC). The main goal of this project is to characterize the ROIC, evaluate it's performance when coupled with GaN sensors, and propose improvements for future ROICs.

\subsection{Gallium Nitride photodiode}\label{ssec:gallium_nitride_uv_sensors}
Avalange photodiodes (APD) are sensitive semiconductor devices that exploit the so called photoelectric effect to convert light into electricity. The photovoltaic effect is the production of free electrons when light is obserbed in a material. If these electrons are created in the instrinsic region of a PN junction, where there is an electric field, a current will be generated. In linear gain mode, the APD is reverse biased by a small voltage. The increase in current is measured and used a measure of light intensity. Another mode of operation is the geiger mode, where the reverse bias voltage is much higher. In this region the APD operates above breakdown. Breakdown is the point where a free electron can gain enough speed to create other free electrons, causing an avalange. The multiplication eventually becomes so high that single photons can be detected. 

One of the big challenges for making UV sensors, is the sensitivity to visible light, which is much more prevalent and acts as a large source of noise. Specifically the sun irradiates a lot more visible light than UV. In order to address this issue, Gallium Nitride is used as a material instead of Silicon. The mean reason being it's large bandgap of $>3\,V$. The energy required to create a free electron in this material is more than the energy carried by a single photon in the visible light spectrum. Therefore Gallium Nitride is intrinsic solarblind. 

However there are downsides to using GaN. GaN materials that can currently be produces contain a lot of defects. This means that the dark current of a GaN APD is orders of magnitude larger than Si APDs. Secondly the breakdown voltage of GaN is much higher than Si at $80-100\,V$ reverse bias. Furthermore this breakdown voltage varies a lot accross different GaN devices due to the variance in the material.

More information on the state of art GaN photodiodes can be found in \cite{nikzad2016single}. The work in \cite{choi2009geiger}, \cite{verghese2001gan} and \cite{carrano2000gan} show the performance of various GaN photodiode designs. 
