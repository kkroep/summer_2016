\documentclass{article}
\usepackage[utf8x]{inputenc}
\usepackage{graphicx}
\usepackage{tikz, amsmath, amssymb, bm, color}
\usepackage{geometry}
\usetikzlibrary{calc}
\usetikzlibrary{shapes,arrows}
\usepackage{todonotes}
\usepackage[american]{circuitikz}
\usepackage{pgfplots}
\usepackage{epstopdf}
\usepackage{listings}
\usepackage{subcaption}
\usepackage{mwe}
\usepackage{float}
\usepackage{cleveref}
\usepackage{ragged2e} % For text alignment


\newenvironment{custom_itemize}{
\begin{itemize}
  \setlength{\itemsep}{0pt}
  \setlength{\parskip}{0pt}
  \setlength{\parsep}{0pt}
}{\end{itemize}}


%\usepackage{multicol} % Used for the two-column layout of the document


%\usepackage{bm}


\title{JPL/CALTECH Interndhip\\ The characetrization of a Readout IC for GaN photodiodes}
\author{Kees Kroep 4246373}


\begin{document}

%  \twocolumn[{%
% \begin{@twocolumnfalse}
  \maketitle
%   \end{@twocolumnfalse}
% }]



\section{General information}
\begin{itemize}
    \item Kees Kroep
    \item Msc Computer Engineering and Msc Micro Electronics
    \item Jet Propulsuon Lab/CALTECH
    \item 17 July to 26 September
    \item Summer Internship
    \item k.kroep1@gmail.com
\end{itemize}

\section{Preparation for your period abroad and contact with your own faculty}
Adfter performing well on the course Digital IC Design in Q2 (2015-2016), prof Dr. Edoardo Charbon invited me to do a project at NASA's Jet Propulsion Lab (JPL) in Los Angeles.\\
\\
The communication progress with JPL was extremely slow, but the main reason is that JPL invites hunderds of student interns to it's facility in the summner. The time it took NASA to approve the project was so long that the project had to be delayed by two weeks. The main problem was that NASA's paperwork was required for me to start the VISA application, which is also a very slow process. \\
\\
The minimum and recommended duration for a summer internship at NASA/JPL is 10 weeks. \\
\\
Both the communication with my faculty and International office where satisfactory.

\section{Study programme}
The programnme I participated in was not a regular study program with courses, but instead a continuation of research that was being done at that time at JPL. A student went on a similar Internship the year before me, where she designed a Readout circuit (ROIC) for a Gallium Nitride photodiode that is currently in development at JPL. My job was to take this ROIC, and characterize it. After characterization I had to combine the ROIC with GaN photodiodes to determnine the performance. Finally I made recommendations for future ROICs and instructions for my successor.\\
\\
The study period of 10 weeks was just right. It is enough time to be able to familiarize yourself with the subject matter and contribute to the research. If the project would be longer it would slow down my study progress. 

\section{Finances}
\Cref{my-label} lists the expenses made during a 10 week stay in Los Angeles. Almost all of them are approximations, but they should be sufficient to give a good incidication of the expenses. For sources of money, I received 2500,- from the Justus van Effen scholarship, and paid the rest with my own savings account, with money that was primarily borrowed from the Dutch governement through study loans.

\begin{table}[H]
    \centering
\caption{Overview of expenses during 10 week stay}
\label{my-label}
\begin{tabular}{l|l}
    \textbf{Expenses} & \textbf{euro}\\ \hline
visa              & 400            \\
flight ticket     & 1300           \\
accommodation     & 2000           \\
food              & 700            \\
travel            & 250            \\
leisure time      & 500            \\ \hline
\textbf{total}    & \textbf{5150} 
\end{tabular}
\end{table}

\section{Accommodation}
Accommodations in Los Angeles are very expensive with prices ranging between 600 and 1200 a month. I had a lot of trouble finding a good accommodation, but I stumbled upon the website places4students.com, which proved to be an invaluable resource to find good accommodations. I settled for a 800\$ per month room in a house where I lived with the landlord. A tip for new students: an unfurnished room is not an option for international students. The lower rent is deceiving because you will pay more in the long run. Especially for short stays, it is not worth the trouble. 

\section{Language and Culture}
Los Angeles is in Americam, and has Hollywood is it, which means that the language is english, and the scene exactly what you will find in Amerixcan movies. There are no hurdles to overcome for this aspect.

\section{Leisure time}
The most surprizing aspect of my stay was that I was introduced to the LA swing dance community. The dance community in Los Angeles is huge, and there are excellent teachers to go with that. My favourite place was Lindy Groove (lindygroove.com), where there are a lot of young people, and a very welcoming ambiance. They have a swingdance evening with lessons and an open social dance every thrursday that I can recommend to anyone who is interested in dancing. \\
\\
Other than that I mostly did activities with interns I met on JPL, like hiking and sightseeing.

\section{Travel}
Traveling is an interesting aspect of Los Angeles. Basically everybody wil tell you that youn can't rely onb the public transport in Los Angeles, and that it is nothing when compared to European countries. Now this is partially true. The public transport is not very reliable, with busses arriving either 5 minutes earkly or 15 minutes late, and very infrequent, with gaps of up to an hour between busses. However, if you don't have a car or drivers license like me, there is not really a choice. I used the opublic transport almost every day, and while the system was lacking and inconvenient at times, it was sufficient for my stay. I would definetly recommend checking out the busses before searching for an accommodation. Specifically the metro bus lines 268 and 177 will take you directly to JPL. Having an accommodation in walking distance to a busstop of onbe of those lines is invaluable. 

\section{Other}
I have several other tips for students going to America. One of them regards the buying of food. There are several big supermarkets to be found everywhere, but some of them are way cheaper than others. Vons, for example, is very expensive, while food4less and Super King are much more affordable. When you buy eggs, make sure to keep them chilled at all times. Americans do weird things with their eggs to combat salmonela, but the shell becomes damaged and fragile. This means that the egg shells lose their natural protective property.

Watch the nutrition labels. I don't do this in the Netherlands but in America it is very necessary. The country is addicted to suger and food manufacturers go to lengths to sneak sugar into their products to sell more.

When walking down the road, keep in mind that California gives a drivers license to every fool with 200 bucks. They will not pay attention and will not see you. Also it is legal to drive through a red light if you turn right. 

Do not work with a salary next to your Internship unless your VISA allows for that. The US Government is known and feared for the handeling of these violations. \\ 
\\
To anyone being fortunate enough to get an internship in Los Angeles or the USA in general: congratulations and I wish you the best of luck for the unforgetable experiences ahead.
\end{document}



